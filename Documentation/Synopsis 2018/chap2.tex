\chapter{Literature Survey}
\section{Summary}
We examine the task of speech to 3D scene generation. There is a myriad of
applications for this technology, mainly for creative and educational industries. Designers can use this technology to interpret and display
their thoughts and imaginations. Students can be taught with a near real-time graphical depiction of the topic. Commercial meetings and conference sessions can make the most of this technology.
\section{Survey}
\paragraph{}The following observations were found in the literature survey:
	\begin{description}
		\item{1.} Text to scene (limited capabilities).
		\item{2.} Limited size databases (no dynamic generation or manipulation).
		\item{3.} Scenes generated are not intelligent and precise hence, cannot be used for real-world applications.
		\item{4.} Language used is unnatural.
	\end{description}
\paragraph{}The following papers were referred and used to understand the current systems and their working, and to derive knowledge of how implementation can be proceeded forward:
	\begin{itemize}
	\begin{flushleft}
		\item{\textbf{Will Monroe, 3D Scene Retrieval From Text With Semantic Parsing}} \hfill \\ \textbf{Learning: } Learnt the concept of semantic parsing from the text.
		\linebreak 
		\item{\textbf{Wordseye:An Automatic Text-To-Scene Conversion System.}} \hfill \\ \textbf{Learning: }Learnt the linguistic analysis of text, and generation of the 3D scene itself.
		\cite{wordsEye}
		\linebreak 
		\item{\textbf{A Supervisory Hierarchical Control Approach For Text To 2D Scene Generation}} \hfill \\ \textbf{Learning: } Detecting changes and positioning of images and scenes.\cite{AI}
		\linebreak
		\item{\textbf{Real-Time Automatic 3D Scene Generation}} \hfill \\ \textbf{Learning: }Determining how to achieve real-time results and how to achieve the ability to detect input in users' natural language.\cite{nlp}
		\end{flushleft}
	\end{itemize}